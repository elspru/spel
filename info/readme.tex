
\documentclass[12pt]{article}

\title{about SPEL Idea}
\author{from Logan Streondj \\
with license CC-BY-SA ya}
\date{time-at \today }
\usepackage{pnets}
\usepackage{pst-tree}
\usepackage{xcolor}
\usepackage{rotating}
\begin{document}

\maketitle
\tableofcontents
\section{Introduction}

about ob contents 
be description and illustration for plural idea part of SPEL ya

about for accessibility 
su all description be superset of illustration yand
su all part of illustration be description in text ya
\section{Motivation}
about now problem
su many programming language be use to program all of computer 
yand su many human language be use to speak 
with all of human ya
su most programming language be design for small part of
computer yand
su most human language for small part of global population ya

about program language example 
su C be use for system program yand su VHDL for hardware program
yand su Javascript for web program yand su SQL for database
program yand su BASh for shell program yand su Java for portable 
program yand su assembly for fast program yand su TeX for print 
document program yand su HTML for web document yand su ChucK for 
audio program yand su SVG for vector graphic yand su OpenCl 
for GPU program yand su many script for small purpose
yand su all of them be have ob different syntax and
style of programming yand su none of them be have ob human
language syntax ya

about SPEL program language solution
be provide ob one core language tha be translate to all other ya

about human language example
su Mandarin be native language for 955 million people or
14\% of earth human yand
su Spanish be native language for 405 million people or 6\% of
earth human yand
su English be native language for 360 million people or 5\% of
earth human ya su other language be have ob less ya

su some be assume ob tha su everyone be wil learn ob English 
but about tha be learn ob new language end-clause be hard ya and
su it be give ob unfair advantage to english speaker ya
%about good example 
%su Indonesia be past have ob majority of Javanese speaker 
%but about tha to avoid give su them ob advantage end-clause 
%su they be choose ob trade language of Malay as official ya

about SPEL human language solution
be provide ob native like language
tha be translate to all other ya

\section{Ideal Flow}
su input agent be have ob idea ya\\
be write ob idea to natural text 
by using word definition of input language tha from SPEL ya\\
be deconjugate ob natural text to analytic input text 
by using conjugation rules of input language ya\\
su SPEL be translate-in ob analytic input text 
to mwak pivot tree 
by using word translation and word order of input language ya\\
be translate-out ob mwak pivot tree to analytic output text
by using  word order and word translation of output language ya\\
be conjugate ob analytic output text 
to conjugated text or source code
by using conjugation rules of output language ya\\
su output agent be compile or understand ob conjugated or source
code to understood idea or native-code by using output word
definition or API library of output language ya\\
su output agent be gain ob skill or app ya

be see ob figure 1 for illustration ya

\psset{arrowlength=2, arrowinset=0, arrowsize=4pt 2}

\begin{figure}
\begin{center}
\begin{petrinet}[xunit=3cm, yunit=2cm](0,0)(4,10)

% language
\psset{linecolor=green}
\source{idea}(0,8) 
\label"{b}idea\strut
\place{inNatural}(0,7) 
\label"{b}natural text\strut
\place{inAnalytic}(0,6) 
\label"{b}analytic input text\strut
\psset{linecolor=black}
\place{pivotLanguage}(0,5) 
\label"{b}mwak pivot tree\strut
\place{outAnalytic}(0,4) 
\label"{b}analytic output text\strut
\place{outConjugated}(0,3) 
\label"{b}conjugated text or source-code\strut
\psset{linecolor=blue}
\place{nativeCode}(0,2) 
\label"{b}understood idea or native-code\strut
\psset{linecolor=black}

% processes
\psset{linecolor=green}
\trans{write}(1,7.5)
\label"{b}be write\strut
\trans{deconj}(1,6.5)
\label"{b}be deconjugate\strut
\psset{linecolor=black}
\trans{transIn}(1,5.5)
\label"{b}be translate-in\strut
\trans{transOut}(1,4.5)
\label"{b}be translate-out\strut
\trans{conj}(1,3.5)
\label"{b}be conjugate\strut
\psset{linecolor=blue}
\trans{understand}(1,2.5)
\label"{b}be understand or compile\strut
\source{use}(1,1)
\label"{b}skill or app\strut
\psset{linecolor=black}

% data
\store{inLibrary}(3,7.75)
\label"{b}input word definition library\strut

\psset{linecolor=green}
\store{inConj}(3,6.75)
\label"{b}input conjugation rules\strut
\psset{linecolor=black}
\store{inDict}(3,5.75)
\label"{b}input word order and word translation \strut

\store{outDict}(3,4.75)
\label"{b}output word translation and word order \strut
\store{outConj}(3,3.75)
\label"{b}output conjugation rules\strut

\store{outLibrary}(3,2.75)
\label"{b}output word definition or API library\strut

% arcs
\psset{linecolor=green}
\arc{idea}{write}
\arc{write}{inNatural}
\arc{inNatural}{deconj}
\arc{deconj}{inAnalytic}
\psset{linecolor=black}
\arc{inAnalytic}{transIn}
\arc{transIn}{pivotLanguage}
\arc{pivotLanguage}{transOut}
\arc{transOut}{outAnalytic}
\arc{outAnalytic}{conj}
\arc{conj}{outConjugated}
\psset{linecolor=blue}
\arc{outConjugated}{understand}
\arc{understand}{nativeCode}
\arc{nativeCode}{use}
\psset{linecolor=black}

\psset{linecolor=green}
\arc{inLibrary}{write}
\arc{inConj}{deconj}
\psset{linecolor=black}
\arc{inDict}{transIn}
\arc{outDict}{transOut}
\arc{outConj}{conj}
\psset{linecolor=blue}
\arc{outLibrary}{understand}
\psset{linecolor=black}

\end{petrinet}
\end{center}
\caption{about ideal SPEL flow be illustration ya
su \textcolor{green}{green} be Input Agent yand
su \textcolor{blue}{blue} be output Agent ya 
su black be provide by SPEL ya}
\label{fig:nodes}
\end{figure}

\newpage
%%%%%%%%%%%%%%%%%%%%%%%%%%%%%%%%%%%%%%%%%5
\section{Ideal Pivot Tree}

su text be usually have ob title and text array ya
su text array be able have ob sentence yand
be able have ob subordinate text or sentence junction ya

su all junction be have ob head and body ya
su all array be able have ob plural member ya

su sentence be usually have ob sentence array and sentence
particle yand be able have ob mood ya
su sentence array be usally have ob  phrase yand
be able have ob top clause or phrase junction ya

su top clause be usually have ob head and subordinate sentence ya

su phrase be usually have ob adposition and type
yand be able have ob subordinate-clause or type junction ya

su subordinate-clause be usually have ob head and 
subordinate sentence ya

su type be usually have ob body yand be able have ob 
body classifier or genitive ya

su genitive be usually have ob head and subordinate phrase ya

be see ob figure 2 for illustration ya


\begin{sidewaysfigure}
\def\dedge{\ncline[linestyle=dashed]}
\begin{psTree}{\Toval{Text}}
\Toval[edge=\dedge]{title}
\begin{psTree}{\Toval{array}}
\Toval[edge=\dedge]{sub Text}
\begin{psTree}{\Toval{Sentence}}
\Toval[edge=\dedge]{mood}
\begin{psTree}{\Toval{array}}
\begin{psTree}{\Toval[edge=\dedge]{topClause}}
\Toval{head}
\Toval{sub Sentence}
\end{psTree} % top clause
\begin{psTree}{\Toval[edge=\dedge]{phrase Junction}}
%\Toval{head}
%\begin{psTree}{\Toval{array}}
%\Toval{plural phrase}
%\end{psTree} % junction array
\end{psTree} % phrase Junction
\begin{psTree}{\Toval{Phrase}}
\Toval{adposition}
\begin{psTree}{\Toval{Type}}
\Toval[edge=\dedge]{classifier}
\Toval{body}
\begin{psTree}{\Toval[edge=\dedge]{Genitive}}
\Toval{head}
\Toval{sub Phrase}
\end{psTree} % genitive

\end{psTree} % type
\begin{psTree}{\Toval[edge=\dedge]{type Junction}}
\end{psTree} % type junction
\begin{psTree}{\Toval[edge=\dedge]{Subclause}}
\Toval{head}
\Toval{sub Sentence}
\end{psTree}  % sub clause
\end{psTree}  % phrase
\end{psTree} % sentence array
\Toval{sentence particle}
\end{psTree} % sentence
\begin{psTree}{\Toval[edge=\dedge]{sentence Junction}}
\end{psTree} % type junction
\end{psTree} % text array

\end{psTree} % text

\caption{about ideal SPEL Pivot Tree be illustration ya
su dashed line be show ob able have yand
su full line be show ob usually have ya}
\end{sidewaysfigure}


\end{document}
