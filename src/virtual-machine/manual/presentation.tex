\documentclass{beamer}
\usepackage{tipa}
\mode<presentation>{
  \usetheme{boxes}
} 
\title{Liberty Bodies: starting with SPEL and GIOS overview}
\author{Logan Streondj \
          (CC-BY-SA)}
\AtBeginSection[]{} % for optional outline or other recurrent slide
\begin{document}
\begin{frame}
\titlepage
\end{frame}

\begin{frame}
\frametitle{Presentation Overview}
  \begin{itemize}
    \item Mission and Vision
    \item SPEL Language 
    \item Machine Intelligence 
    \item Market Overview
    \item Funding Model
    \item Governance
    \item Use Cases
  \end{itemize}
\end{frame}

%fall rise fall rise format

\begin{frame}
  \frametitle{Singularity is Coming!}
  Imagine incarnating as a proprietary robot.
  \begin{itemize}
    \item Human level AGI and robots will become available for reincarnation.
    \item Big proprietary companies like Microsoft, Google, or Apple may do it.
    \item If you incarnate in a proprietary host body.
    \item you'll be dependant on the manufacturer for parts.
    \item vulnerable to obsolescence, if manufacturer stops making them.
    \item forced to pay licensing fees simply to continue activation. 
    \item having minimal to no control over your own software. 
    \item a perpetual life of slavery with little hope of escape. 
  \end{itemize}
\end{frame}

\begin{frame}
 \frametitle{Mission Statement}
 A Libreware human-incarnation-worthy General-Intelligence Operating-System (GI-OS).
 \begin{description}
    \item[libreware] Open-source software and hardware
    \item[incarnation-worthy] Supporting at least the homo-sapien feature set. 
    \item[General-Intelligence] AI which is adaptable to new situations. 
    \item[Operating-System] Vertical-integration from lowest layers up. 
  \end{description}
\end{frame}


\begin{frame}
  \frametitle{Human Babel}
  \begin{itemize}
    \item there are over 7,000 languages in the world.
    \item 400 of them have over a million speakers. 
    \item statistical machine translation is best for a gyst.
    \item lack of interoperability favours language extinction. 
    \item high precision ubiquitous translation can allow diversification.
  \end{itemize}
\end{frame}

\begin{frame}
  \frametitle{Computer Babel}
  \begin{itemize}
    \item For operating a human body only need one native language
    \item For operating a computer have dozens of programming langauges
          and non-compatible protocols (mini languages).
    \item number of architectures and instructions is increasing. 
    \item The number of programming paradigms or dialects is increasing,
          bloating many of the languages that support them.
    \item Many popular language features are not parallel friendly,
          (Globals, Recursion, OO) learning them is couter-productive.
    \item Most languages are cache opaque, increasing data-bottlenecks.
    \item learning and reimplementing in  many changing languages 
          for various architectures wastes programmer time. 
  \end{itemize}
\end{frame}

\begin{frame}
  \frametitle{Idle Static Computers}
  \begin{itemize}
    \item Computers are often idle when not being used. 
    \item They do not program, debug and test to improve user experience. 
    \item Humans spend more time programming, than computers do compiling. 
    \item GPU's have most processing power but often not used by compilers.
    \item When my computer is idle, I feel like I'm not getting my money's worth.
  \end{itemize}
\end{frame}

\begin{frame}
  \frametitle{Vision}
  Imagine incarnating as a Libreware robot. 
  \begin{itemize}
    \item Do you own your software? Yes. Can you modify it? Yes.
    \item Do you own your hardware? Yes. Can you modify it? Yes.
    \item How many programming languages do you need to know to manage your
software and hardware? One.
  \end{itemize}
\end{frame}

\begin{frame}
  \frametitle{One Lang to Rule Them All!}
  Vision is to have all human brains capable of contributing to GI-OS,
  regardless of their native language, or langauge preference. 
  \begin{itemize}
    \item from machine code to human communication in one langauge, Pyash.
    \item using most common grammar and vocabulary of world languages. 
    \item easily translates between formal variants of all human languages.
    \item designed to be easily portable to GPU's, FPGA's and QPU's.
  \end{itemize}
\end{frame}

\begin{frame}
  \frametitle{Pyash Phonology}
  \begin{itemize}
    \item 22 consonants, and 6 vowels. most common phonemes on phoible.org 
    \item i/i/, a/\textipa{"a}/, u/u/, e/\textipa{E}/, o/\textipa{O}/, 6/\textipa{@}/
    \item b/b/, c/\textipa{S}/, d/d/, f/f/, g/g/, j/\textipa{Z}/, k/k/, l/l/, m/m/, n/n/, p/p/, q/ŋ/, r/r/, s/s/, t/t/, v/v/, w/w/, x/x/, y/j/, z/z/.
    \item 2 syntax consonants "."/\textipa{P}/ and "h"/h/
%   \item 2 tones high "7"/\tone{44}/ and low "\_"/\tone{22}/
    \begin{itemize}
        \item about 25\% of languages have extensive tonal vocabualary,
        \item 70\% are at least partially tonal,
        \item all languages use intonation. 
    \end{itemize}
    \item 2 local definition consonants "1"/|/ "8"/\textipa{||}/
    \item 2 root word types HCVC and CCVC
    \item 2 grammar word types CV and CCVH
  \end{itemize}
\end{frame}

\begin{frame}
  \frametitle{Pyash Grammar}
  \begin{itemize}
    \item Grammar based largerly on WALS (World Atlas of Language Structures)
    \item SOV postpositional, like Japanese and majority of languages.
    \item in Pyash: hwacwu mina prumka hcotli
    \item in Analytic English: hey world su me be enjoy ob computer-programming really 
    \item in Conjugated English: O world I computer-programmingan enjoyeth. 
  \end{itemize}
\end{frame}

\begin{frame}
  \frametitle{Sample Phrases}
  \begin{itemize}
    \item hpepna hyunka coli 
         (The apple is red.)
    \item yana .djon.giye hpepka coli 
         (It is John's apple.)
    \item mina .djon.giyi hpepka kcinli 
         (I give John the apple.)
    \item yapina .djon.giyi kcinli 
         (he gives it to john.)
    \item yajina yapiyi kcinli 
         (she gives it to him.)
    \item nyahna koyapiyi hpepka kcinlaka twinli 
         (We want to give him the apple.)
  \end{itemize}
\end{frame}

\begin{frame}
  \frametitle{Pyash Vocabulary: Phonological Makeup}
  \begin{itemize}
    \item attempted manual creation, error prone and cumbersome.
    \item generated all legal easy to pronounce words
    \item made script to harvest words from 30+ most common languages.
    \item got phonological data for words using espeak and some scripts.
    \item generated all words using those phonemes and assigned weights.
    \item weighted by populations representing major language families.
  \end{itemize}
\end{frame}

\begin{frame}
  \frametitle{Pyash Vocabulary: Content Extent}
  \begin{itemize}
    \item made script that
    \item got together major "international" word-lists
    \item wordnet, oxford-3000, special-english, among others ~10k total
    \item sorted by 30,000 word frequency list of gutenberg material
    \item removed ambigious words with algorithms and blacklists
    \item removed words which have same meaning as previously defined word in
          any of the 40+ major languages.
    \item removed words which were over borrowed
    \item thus each root word is an orthogonal part of semantic space.
    \item generates a thesaurus style dictionary, and suggest list for all
          languages.
  \end{itemize}
\end{frame}

\begin{frame}
  \frametitle{Pyash Vocabulary: Numbers and Encoding}
  \begin{itemize}
    \item 16 bit encoding, 65,536 numbers.
    \item ~32,000 encodable words, encoding extendable to 57,000 words.
    \item ~18,600 easily pronounceable (legal) words.
    \item each Pyash root word is a word family.
    \item ~2,500 root words in seed vocabulary, based on ~8,000 words. \\
          Includes oxford-3000, wordnet-core and special-english among others. 
    \item 3,000 word families and top 5,000 words enough for 95% of word usage in
          English.
    \item ~4,000 root words in medium vocabulary, based on ~11,500 words. \\
          Includes top 5,000 words.
    \item 5,000 word families enough for 99.9% of word coverage,
    \item ~8,000 root words in giant vocabulary, based on ~39,000 words. 
    \item native level fluency estimated to be 20,000 to 40,000 words. 
  \end{itemize}
\end{frame}

\begin{frame}
  \frametitle{Pyash Vocabulary: future}
  \begin{itemize}
    \item definitions can be sourced from OmegaWiki
    \item people will be able to add words and fix translations
    \item root words are mostly stable, 
    \item some grammar words changing for use in virtual machine
    \item 1.0 will be a long term support vocabulary release. 
  \end{itemize}
\end{frame}

\begin{frame}
  \frametitle{Implementation History}
  \begin{itemize}
    \item initial parser written in Haskell ~2006 for simplified Lojban
    \item wrote parser in Java ~2008 called Rpoku
    \item wrote interpreter in x86 nasm assembly based on Forth 2011-2014
    \item wrote compiler in Javascript in 2014-2015
    \item writing virtual machine in OpenCl compatible C, that compiles to JS
          (2016)
  \end{itemize}
\end{frame}

\begin{frame}
  \frametitle{Machine Programmer}
  Because tired of coding and recoding, and because of marked success of
  vocabulary generation, making a machine programmer to code and recode the
  libraries. 
  \begin{itemize}
    \item register based virtual machine code is subset of Pyash.
    \item starts by evolving simple programs via genetic programming
    \item later can make more complete programs with more elaborate machine
          intelligence, such as deep neural nets, and unsupervised learning. 
    \item all generated programs are valid Pyash, so human speakable. 
    \item thus Machine Programmer is also "fluent" in Pyash. 
  \end{itemize}
\end{frame}


\begin{frame}
  \frametitle{Machine Programmer: Components}
  Programming with a machine programmer.
  hbuc for user, and mlic for machine.
  tləh remote-future-tense, glah soon-future-tense, wi future-tense, 
  ra present-tense.
  \begin{itemize}
    \item trouble identify (hbucra mlictləh) proprietary bodies
    \item answer speculate (hbucra mlictləh) libreware bodies
    \item natural language program specification (hbucra mlictləh) liberty bodies
    \item program dissection into branch programs (hbucra mlicwi) SPEL then
        GI-OS then open hardware then interplanetary colony.
    \item input and produce selection for branch programs (hbucra mlicwi) input
English word output Pyash word.
    \item quiz generation for branch programs (hbucra mlicglah) if user then
hbuc.
    \item ingredient selection for  branch programs (hbucra mlicglah)
comparison, loop, conditional
    \item write branch programs to fit quizes (mlicra)
    \item repair or modernize programs for new specifications (mlicra)
    \item compile byte-code to machine code (mlicra)
    \item adapt virtual-machine to new architectures (hbucra mlicwi)
  \end{itemize}
\end{frame}

\begin{frame}
  \frametitle{Libreware Poverty}
  \begin{itemize}
    \item Most libreware projects receive little or no money.
    \item Some companies spend large sums on Libreware, but pay their own
          developers, rather than the projects
    \item most consumers can't buy developers to add features or fix bugs in
          libreware software. And can't add/fix them themselves. 
    \item As a result libreware does best in mature markets, 
          where it can undercut existing proprietary solutions. 
    \item providing support services seems to be best revenue model. 
  \end{itemize}
\end{frame}

\begin{frame}
  \frametitle{Monetization: Market Size}
  \begin{itemize}
    \item Machine Translation, ~360 mln, 0.6\% slow growth
    \item Intelligent Virtual Assistants, ~580 mln "rapid growth"
    \item Packaged Software ~430 billion, ~5\% growth, moderate growth.
    \item Software Development (china), ~690 bln, 21.4\% rapid growth.
  \end{itemize}
\end{frame}

\begin{frame}
  \frametitle{Platforms}
  Mature can be relied on. 
  \begin{itemize}
    \item Server*
    \item Desktop*
    \item Peripherals (Keyboard/Mouse/Joystick)
  \end{itemize}
  Growing can be innovated with. 
  \begin{itemize}
    \item Mobile
    \item Voice*
    \item Virtual Assistants*
    \item Virtual Reality
    \item Augmented Reality (Pokemon Go)
  \end{itemize}
  * Pyash targets
\end{frame}

\begin{frame}
\frametitle{Potential Competitors}
  \begin{itemize}
    \item IBM Watson (cost ~30-50 mln to develop)
    \item Google Brain (Deep mind bought for 500 mln), 
            Human Brain Project (1bln funded)
    \item Cortana, Alexa, Siri, Viv
    \item Upwork (1 bln revenue), Freelancer
  \end{itemize}
\end{frame}

\begin{frame}
\frametitle{Potential Allies}
  \begin{itemize}
    \item OpenAI (1 Bln funded), OpenCog
    \item Linux/Minix/FreeRTOS
    \item GNU/FSF, Ubuntu
    \item BOINC grid computing network
    \item Adapteva, lowRISC (open hardware)
    \item FreedomSponsors (source code)
    \item Python machine learning libraries
    \item the 75%+ of the world population not fluent in English.
  \end{itemize}
\end{frame}

\begin{frame}
  \frametitle{SPEL/GI-OS Cryptocurrency Supermarket}
  Vision is that users can make feature requests and put coin rewards on them, 
  which would then be rapidly solved by various machine programmers and
  human-machine programmer teams, which want the coins. 
  \begin{itemize}
    \item machine programmer supermarket for Liberty-bodies project
    \item most available coins are feature rewards
    \item 1/13 of coins each for welfare waterfall, auction and web servants
    \item percentage of earned reward must be reinvested charitably, 
          as rewards to other parts of the project -- charity has been shown to
help increase peoples happiness more than actually receiving. 
    \item prices can be pegged on average energy requirements in joules, or
          work-coin. based on joules expended plus a premium of 60-85%,
          80% premium equivalent to a school teacher in a western country. 
    \item Initial market cap of 50 exa-joule (biggest
highly-composite number to fit in 64bits) or about 1 trillion dollars 
          If calculated at residential kw/h prices. Approximate GDP of Mexico.
    \item 1 trillion dollars is enough to pay 33 million basic incomes \$15k/yr,
          though coin will only become available as features are added and bugs
are fixed. 
  \end{itemize}
\end{frame}

\begin{frame}
  \frametitle{Virtual Assistants}
  Money is just representative of energy for motivation. People are who get
things done, so forming relationships is most important. 
  \begin{itemize}
    \item GI-OS installations could educate their users
    \item their underlying purposes would be for Libreware bodies.
    they may motivate their users in that direction. 
    balanced with the purposes of the user. 
    \item various purposes for the user could be pre-programmed as many don't know
what they want, in those cases it would be happiness via healthy lifestyle, 
    good relationships, events and vistas.   
    \item can help user achieve Purpose, Domain and Liberty
    \item Purpose by seeing how the user can fit into the big picture of Liberty
bodies. 
    \item Domain by finding which beneficial skills the user wishes to acquire,
      and supporting them in achieving those skills. 
    \item Liberty, such as financial liberty, by helping with income streams, 
          budgeting, and taxes.
  \end{itemize}
\end{frame}

\begin{frame}
  \frametitle{Governance Issues: DAO motivation}
  \begin{itemize}
    \item Humans can only form relationships with up to 200 people,
          due to cerebral cortex limitations. 
    \item Benevolent dictatorships can lead to stagnation and fundamentalism. 
    \item Humans are prone to corruption via bribes and favours. Paying people
more than their fair share devalues the currency. 
    \item Material rewards lower cognitive performance of humans yet 
          C-class positions often get huge bonuses.
    \item Centralized organizations can be stopped by seizing of servers, 
          or the main organizers. Or even their absenteeism. 
  \end{itemize}
\end{frame}

\begin{frame}
  \frametitle{Governance: DAO model}
  Ideally SPEL/GI-OS will be a Distributed Automated Organization, so it needs
  to have an automated executive.
  \begin{itemize}
    \item President or "speaker of the house" follows parliamentary authority 
          to manage meetings.  
    \item Secretary records meetings and adjusts company policy according to
          passed amendments.
    \item Treasurer allocates funds for accepted projects. 
    \item Officers enforce policies within company jurisdiction (marketplace).
    \item Comittees research, formulate and test policy ideas. 
    \item Automated government is not prone to bribery or "lobbying". 
    \item Hacks are possible, so continuous open source security analysis
           and patching necessary.  Corruption and compromise is manageable.
    \item direct and-or liquid deliberative democracy is achievable. 
  \end{itemize}
\end{frame}

\begin{frame}
  \frametitle{Use Cases: Governance}
  Holding companies such as Disney, Alphabet and Kraft, 
  or companies that buy other companies and manage them,
  are generally the most lucrative businesses. They can have huge profit
  margins, in excess of even 100%. And are potential customers. 
  \begin{itemize}
    \item A robot representative can take all their constituents opinions 
          into account, when voting on or speaking about an issue.   
    \item GI-OS with governance is ideal for managing multi-lingual companies
          and countries. 
    \item Replacing C-class positions with robots lowers need for bonuses, and
          thus increases shareholder dividends. 
    \item European Union spends over 300 million Euro on translation costs
          alone. And the "representatives" aren't elected, so their motives are
          dubious at best. Replacing them with robots may be best.
  \end{itemize}
\end{frame}


\end{document}
