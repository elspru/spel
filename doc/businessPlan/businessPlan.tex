\documentclass{article}
\usepackage{utf8}
\title{Speakable Programming for Every Language\\
Business Plan}
\author{Logan Streondj}
\begin{document}
% cover page
\maketitle
% and table of contents
\tableofcontents
% Algorithm
%

% executive summary
% or abstract

\section{Playing Field}
% 0 the playing field                                              
%* industry background
\subsection{Industry}
ISIC 8549 Other Education -- language instruction.
NAICS 611420 Computer Training: computer programming school.
%* market analysis
%* business environment analysis
%* Who are the company's customers, and how will the company
%*market and sell its products to them?
%* What is the size of the market for this solution?

\section{Problem}
% 1 the problem                                                    
%*    What problem does the company's product or service solve?
%* What niche will it fill?
%* competitor analysis
%* Who are the competitors
\subsection{Competitors}
\subsubsection{STE: Simplified Technical English}
\subsubsection{Gellish: product modeling}
\subsubsection{SBVR: Semantics of Business Vocabulary and
Business Rules}

\subsubsection{BASIC}
\subsubsection{Smalltalk}
\subsubsection{Oracle University}
Oracle Certification Programs
\subsubsection{CISCO Career Certification.}
%* how will the company maintain a
%*competitive advantage?
%* SWOT analysis
\subsubsubsection{Strengths}
Well known and established.

Smalltalk has few levels to reach the full language:
Smalltalk begins with Scratch in preschool, 
followed by Squeak in middle school and onward.

\subsubsubsection{Weaknesses}
education has a late start.
STE, Gellish and SBVR are only applicable to adults.
Oracle and CISCO only target high school and above.
Microsoft Small Basic starts in middle school.

it is broken into different languages.
Oracle's Java for instance,
starts in high-school as Greenfoot,
followed by BlueJ in early university, 
followed by Netbeans, followed by Java.

BASIC starts with
Microsoft Small Basic, then Full Basic, then Visual Basic.
Then the somewhat unrelated C#.



%* What are the risks and threats confronting the business, and
%*what can be done to mitigate them?

\section{Solution}
% 2 the solution                                                   
%* What is the company's solution to the problem?
%* mission statement
%* business description
\subsection{Competitive Advantage}
Programming Education starting from infancy with board books.
Using the same syntax and language at all levels.


%* marketing plan
%* operations plan
%* What is the business model for the business (how will it
%*make money)?
%** transforming business logic
%** training certification
%** bounties
%*** forward bounties
%*** reverse bounties
%* dual-licensing with cost per proprietary item sold. 


\section{Projections}
% 3 the projections   
%* financial plan
%* What are the company's capital and resource requirements?
%* What are the company's historical and projected financial
%*statements?
%* attachments and milestones
%* management summary
%* How does the company plan to manage its operations as it
%*grows?
%* Who will run the company and what makes them qualified to do
%*so?





\end{document}
